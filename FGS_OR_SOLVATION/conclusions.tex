A dielectric continuum model based on a definition of the dielectric
permittivity as a smooth function of the electron density was applied to the
calculation of specific rotation at the CCSD level for molecules in solution.
For (\emph{S})-methyloxirane, the computed ORD curves for most of the polar
solvents are qualitatively in agreement with experimental measurements,
but the CCSD results with the FGS solvent model failed to predict the
correct positive sign of the molecule in water. Furthermore, the implicit
model was unable to correctly predict the experimentally observed variation 
in nonpolar solvents with comparable dielectric constants. The failure
of the smooth implicit model considered here in conjunction with CCSD
is in line with previous attempts to model the specific
rotation for methyloxirane in solution at lower levels of theory, and it
appears an explicit consideration of specific solute-solvent
interactions may be unavoidable for accurately modeling these effects.

In two other chiral molecules considered here, (\emph{1S},\emph{4S})-norbornenone
and (\emph{S})-2-chloropropionitrile, CCSD specific rotations computed in
solution via the FGS continuum model not only failed to satisfactorily
reproduce the solvent shifts seen in experiment, but also demonstrated
poor performance relative to contemporary continuum solvent models, in particular the fully coupled
CCSD-PCM method. Previously applied to computing excitation energies in solution
with success at the EOM-CCSD level, the failure of the CCSD FGS calculations
here is attributed to the lack of solvent response in the PTE-like treatment
of the solvent for post-HF calculations. Similar observations have been
made when incorporating solvent effects through embedding potentials, and
the specific rotations computed here demonstrate that the contribution of solvent
response may be large enough at times to account for the majority of observed
specific rotation solvent shifts.

