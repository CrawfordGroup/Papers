Specific rotation values were computed with the CCSD method for three
molecules, (\emph{S})-methyloxirane, (\emph{1S,4S})-norbornenone, and
(\emph{S})-2-chloropropionitrile using the aug-cc-pVDZ basis set.
\cite{Kendall:92}
Each of these calculations utilize
the velocity-gauge representation of the dipole operator, ensuring
origin-independent results, and all reported specific rotation values represent
modified velocity-gauge results.\cite{Pedersen:04b}
Each of these geometries was optimized with the B3LYP DFT functional
using the 6-311G(d,p) basis set.\cite{Krishnan:80} The core orbitals
(i.e., 1$s$ for C,N, and O; $1s2s2p$ for Cl) were kept frozen
in the CCSD energy and response calculations.
In defining the dielectric permittivity according to Equation \ref{eqn:eps},
the solvent cavity was fixed based on a converged electronic SCF density
computed in vacuum, as in Reference \citenum{Howard:17}. For selecting
the dimensions of the real-space grid used in solving the GPE for the
electrostatics in solution, the number of points and grid spacing were
selected such that specific rotations computed in vacuum with electrostatic
computed from the GPE reproduced those from a conventional CCSD specific
rotation calculation. For the three molecules studied here, the maximum
deviation between those rotation values is less than 4 \rotunits, and
corresponds to the specific rotation of (1\emph{S},4\emph{S})-norbornenone
at 355 nm, which has a magnitude greater than 3000 \rotunits. The values of the electrostatic potential on the boundary points of the grid are approximated by computing the potential associated with a homogeneous medium of $\epsilon$ corresponding to the solvent's dielectric constant, as in previous works.\cite{Dziedzic:11,Howard:17} The grid
details, as well as parameters used in the DL\_MG multigrid calculations,
are provided in the Supporting Information. For the PCM calculations performed
here, geometry optimizations were performed with Gaussian09 software
\cite{g09}. All other computations were performed with the Psi4 software
package,\cite{psi4} utilizing an interface to PCMSolver\cite{pcmsolver}
for specific rotation calculations with PCM solvents. Those calculations
utilized PCM cavities constructed with Bondi radii scaled by a factor of 1.2.
