Successfully modeling chiroptical properties such as optical rotation
ostensibly requires the same
theoretical considerations involved in modeling any molecular property,
e.g., choices in atomic orbital (AO) basis sets and approaches to electron
correlation effects. Optical rotation calculations, however, demonstrate
a particular sensitivity to these theoretical details and others,
including the treatment of
the environment surrounding a chiral molecule, and represent an ongoing challenge
to theoretical quantum chemistry.\cite{Crawford:06,Srebro:17,Crawford:18}
Highly correlated quantum mechanical
(QM) methods\cite{Shavitt:09,Crawford:00,Helgaker:12} offer the most reliable means of calculation, however
even an accurate correlated
method such as the coupled cluster singles and doubles (CCSD) method
can produce specific rotation values significantly at odds with experimental
results.
One difficulty in comparing accurate QM wavefunction methods with
experiment is that especially large, flexible AO basis sets
are needed to obtain converged optical rotation values near the complete basis
set (CBS) limit,\cite{Mach:11,Baranowska:13,Haghdani:16} and, in contrast to many
molecular properties,\cite{Papajak:11} diffuse basis functions are of
critical importance in optical rotation calculations.\cite{Baranowska:10,
Mach:11,Wiberg:13}
Even with a sufficient treatment of dynamical electron correlation and a basis set
approaching the CBS limit, a single optical rotation calculation on
a molecular system in its minimum-energy geometry may be an inappropriate 
comparison to an experimental value. For example, it is known that
vibrational corrections to specific rotation can be significant,\cite{Ruud:01,
Ruud:05,Mort:05,Kongsted:08,Pedersen:09,Pedersen:09b} even to the
extent that the molecular vibrations induce sign changes at points in the optical
rotatory dispersion (ORD) curve.\cite{Ruud:05}
For conformationally flexible molecules, a great number of conformations
may need to be considered
depending on the experimental conditions.\cite{Wiberg:03,Wiberg:05,
Lambert:12,Pearce:17}

In addition to these factors, the role of solvent is understood to be
of primary importance in ORD spectra.\cite{Kumata:70,Berova:00}
The specific rotation of a molecule measured in the gas phase can differ
drastically from the rotation in solution or the condensed phase,
even changing sign depending on the solvent.\cite{Kumata:70,Wilson:05}
In contrast to many other properties where nonpolar solvent environments
are expected to more closely mimic conditions in the gas phase
relative to polar solvents, the opposite can be true for measured specific
rotations.\cite{Wilson:05} An additional complication in modeling
rotation in solution is that the structure of the surrounding solvent molecules
itself may also contribute to specific rotation, as the ``chiral
imprint'' can be responsible for a significant portion of the solvent
shift.\cite{Beratan:06,Beratan:07} A more complete understanding of the
nature of these solvent effects 
is essential to increasing the reliability of theoretical
optical rotation predictions across all phases.

Solvent effects have been incorporated into optical rotation calculations
often with an implicit solvation approach, such as the
venerable polarizable continuum model (PCM).\cite{Miertus:81,Tomasi:05}
Often used in conjunction with density functional theory (DFT) methods,
DFT/PCM treatments have yielded mixed results in specific rotation
calculations in comparison to experimental values.\cite{Mennucci:02,
Kongsted:08,Lahiri:13}
An explicit treatment of solvent molecules has the advantage of capturing
specific solute-solvent interactions inaccessible to a continuum model,
(e.g., hydrogen bonding). The inclusion of explicit solvent effects
has contributed valuable insight into the role of solvent molecules
in methyloxirane's optical rotation, probing the contributions of
the dissymmetric solvent structure,\cite{Beratan:06,Beratan:07}
as well as achieving quantitative agreement with experimental measurements
in water when combined with an implicit treatment of the bulk solvent
and vibrational corrections.\cite{Lipparini:13} Unfortunately, 
explicit solvation treatments require extensive sampling of configurations, 
and typically thousands of molecular dynamics snapshots are needed for converged
results.\cite{Beratan:06,Beratan:07,Lipparini:13} The associated computational
cost of of such an approach necessarily limits the applicability of most
wavefunction-based electronic structure methods, and one often relies on
DFT methods, where gas-phase specific rotation results, for methyloxirane
in particular, depend on significant error cancellation to obtain correct
signs.\cite{Kongsted:06} In this work, we investigate the combination of
coupled-cluster linear response theory and an implicit solvation model
based on a smooth definition of the dielectric permittivity to obtain
specific rotation values in solution.

The implicit solvent model utilized herein is based on a definition
of the dielectric permittivity as a function of electron density. 
Introduced by Fattebert and Gygi\cite{Fattebert:02,Fattebert:03},
this model was developed for use in the context of plane-wave based DFT
calculations
\cite{Scherlis:06,Andreussi:12,Dziedzic:11,Dziedzic:13,Fox:14}
and has been particularly successful in computing accurate solvation
energies relative to contemporary solvent models.\cite{Dziedzic:11,
Dziedzic:13}
The definition of the dielectric permittivity takes the form of Equation
\ref{eqn:eps}.

\begin{align}
\epsilon[\rho({\mathbf{r}})] = 1 + \frac{\epsilon_\infty-1}{2}\left[1+\frac{1-\left(
\rho({\mathbf{r}})/\rho_0\right)^{2\beta}}{1+(\rho({\mathbf{r}})/\rho_0)^
{2\beta}}\right]
\label{eqn:eps}
\end{align}
In this model, the dielectric permittivity $\epsilon$ is determined by the
electron density
($\rho$) and depends on the two parameters $\rho_0$ and $\beta$, which
control the transition of $\epsilon$ from the vacuum value of one to the
dielectric constant of the bulk solvent $\epsilon_\infty$. The appropriate electrostatic problem
in this case corresponds to solving the generalized Poisson equation (GPE)
(Equation \ref{eqn:GPE}) for the electrostatic potential ($V^\mathrm{GPE}$).

\begin{align}
\nabla \cdot  \left[ \epsilon({\mathbf{r}})\nabla V^\mathrm{GPE}({\mathbf{r}})\right] = 
-4\pi \rho_\mathrm{tot}({\mathbf{r}})
\label{eqn:GPE}
\end{align}
Unlike the standard Poisson equation corresponding to electrostatics in vacuum
or a homogeneous permittivity, which can be solved analytically,
the GPE in Equation \ref{eqn:GPE} must in general be solved by numerical
techniques. Its solution here is achieved by interfacing with 
DL\_MG,\cite{Womack:18} an open-source multigrid solver library. As previously
described,\cite{Howard:17} real-space grid representations of the permittivity
and total density $\rho_\mathrm{tot}$ are used to obtain $V^\mathrm{GPE}$,
which is numerically integrated into the AO basis and iterated to self-consistency
along with the HF molecular orbitals.
We have previously\cite{Howard:17} incorporated this continuum
solvent model in a Hartree-Fock framework to extend its use to
correlated wavefunction methods.
A study of small molecules with well-known electronic excitation
solvent shifts demonstrated good performance for this solvation model
in conjunction with the EOM-CCSD method, typically reproducing experimentally
determined n$\rightarrow \pi^*$ blue shifts to within 0.1 eV.\cite{Howard:17}
This continuum dielectric model requires only two parameters and,
capturing the effects of solvation through the molecular orbitals
and Fock matrix elements (analogous to a PTE approximation in PCM)
\cite{Lipparini:16,Cammi:13},
can be readily extended to any post-HF
correlated method.  This approach can be contrasted with a more complete
treatment of continuum solvation at the correlated level, which
would include the solvent's response to electron correlation and/or the
electric and magnetic field perturbations, as well as a proper
consideration of non-equilibrium solvation effects, as has been demonstrated
by Caricato.\cite{Caricato:13} Despite the present model's relative simplicity,
excitation energies computed via EOM-CCSD calculations in solution have
compared favorably with those resulting from more sophisticated approaches.
\cite{Howard:17} In this work, we apply this smooth dielectric model
(hereafter, referred to as the FGS model)
to the computation of specific rotations in solution at the CCSD level.

